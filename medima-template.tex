%% This is file `medima-template.tex',
%% 
%% Copyright 2018 Elsevier Ltd
%% 
%% This file is part of the 'Elsarticle Bundle'.
%% ---------------------------------------------
%% 
%% It may be distributed under the conditions of the LaTeX Project Public
%% License, either version 1.2 of this license or (at your option) any
%% later version.  The latest version of this license is in
%%    http://www.latex-project.org/lppl.txt
%% and version 1.2 or later is part of all distributions of LaTeX
%% version 1999/12/01 or later.
%% 
%% The list of all files belonging to the 'Elsarticle Bundle' is
%% given in the file `manifest.txt'.
%% 
%% Template article for Elsevier's document class `elsarticle'
%% with harvard style bibliographic references
%%
%% $Id: medima-template.tex 153 2018-12-01 11:38:32Z rishi $
%% $URL: http://lenova.river-valley.com/svn/elsarticle/trunk/medima-template.tex $
%%
%% Use the option review to obtain double line spacing
%\documentclass[times,review,preprint,authoryear]{elsarticle}

%% Use the options `twocolumn,final' to obtain the final layout
%% Use longtitle option to break abstract to multiple pages if overfull.
%% For Review pdf (With double line spacing)
%\documentclass[times,twocolumn,review]{elsarticle}
%% For abstracts longer than one page.
%\documentclass[times,twocolumn,review,longtitle]{elsarticle}
%% For Review pdf without preprint line
%\documentclass[times,twocolumn,review,nopreprintline]{elsarticle}
%% Final pdf
\documentclass[times,twocolumn,final]{elsarticle}
%%
%\documentclass[times,twocolumn,final,longtitle]{elsarticle}
%%


%% Stylefile to load MEDIMA template
\usepackage{medima}
\usepackage{framed,multirow}

%% The amssymb package provides various useful mathematical symbols
\usepackage{amssymb}
\usepackage{latexsym}

% Following three lines are needed for this document.
% If you are not loading colors or url, then these are
% not required.
\usepackage{url}
\usepackage{xcolor}
\usepackage{booktabs}
\usepackage{hyperref}

\definecolor{newcolor}{rgb}{.8,.349,.1}

\journal{Medical Image Analysis}

\begin{document}

\verso{Given-name Surname \textit{et~al.}}

\begin{frontmatter}

\title{Type the title of your paper, only capitalize first
word and proper nouns\tnoteref{tnote1}}%
\tnotetext[tnote1]{This is an example for title footnote coding.}

\author[1]{Given-name1 \snm{Surname1}\corref{cor1}}
\cortext[cor1]{Corresponding author: 
  Tel.: +0-000-000-0000;  
  fax: +0-000-000-0000;}
\author[1]{Given-name2 \snm{Surname2}\fnref{fn1}}
\fntext[fn1]{This is author footnote for second author.}
\author[2]{Given-name3 \snm{Surname3}}
%% Third author's email
\ead{author3@author.com}
\author[2]{Given-name4 \snm{Surname4}}

\address[1]{Affiliation 1, Address, City and Postal Code, Country}
\address[2]{Affiliation 2, Address, City and Postal Code, Country}

\received{1 May 2013}
\finalform{10 May 2013}
\accepted{13 May 2013}
\availableonline{15 May 2013}
\communicated{S. Sarkar}


\begin{abstract}
%%%

%%%%
\end{abstract}

\begin{keyword}
%% MSC codes here, in the form: \MSC code \sep code
%% or \MSC[2008] code \sep code (2000 is the default)
\MSC 41A05\sep 41A10\sep 65D05\sep 65D17
%% Keywords
\KWD Keyword1\sep Keyword2\sep Keyword3
\end{keyword}

\end{frontmatter}

%\linenumbers

%% main text
\section{Introduction}
\label{sec1}

%\section*{Acknowledgments}
%Acknowledgments should be inserted at the end of the paper, before the
%references, not as a footnote to the title. Use the unnumbered
%Acknowledgements Head style for the Acknowledgments heading.
%%Harvard
\bibliographystyle{model2-names.bst}\biboptions{authoryear}
\bibliography{refs}
\end{document}

%%
